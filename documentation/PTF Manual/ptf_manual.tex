\documentclass[a4paper,12pt]{article}

\usepackage{times,latexsym,epsfig,epsf,graphicx,appendix,lineno, cite}
%\usepackage{times,latexsym,epsfig,epsf,graphicx,appendix,lineno}
\usepackage{amssymb}
%\usepackage[pdftex]{graphicx}%
\usepackage{textcomp}
\usepackage{url}
\usepackage{multirow}
\usepackage{amsmath}
\usepackage{hyperref}
%\setlength{\textheight}{8.5in}
%\setlength{\textwidth}{6.5in}
%\def\imagetop#1{\vtop{\hbox{#1}\null}}
\usepackage[margin=2.5cm]{geometry}


\begin{document}

\title{
PMT Test Facility User Manual}

\maketitle

\section{Introduction}

The following is meant to provide sufficient instructions for the user to be able to operate the PMT 
test facility (hereafter PTF).

\section{Motor Control}

There are a set of 10 motors for the PTF, 5 for each gantry; they are 3 translation motor, 1 rotation motor and 
1 tilt motor.

The motor are driven by a pair of Galil Digital Motor Controllers, which are mounted on electronics rack.

\subsection{Limit Switchs}

For the three translation axes we have mechanical limit switches. The output of the limit switches are
attached to the Galil controller; if a gantry reachs a limit switch motion on that axis (in a particular direction)
will be immediately stopped.

For the rotation axis we have a pair of magnetic limit switchs of the rotary table.  Again, these limit
switches will prevent motion beyond a certain angle.

We do not have any electrical limit switches for the tilt axis.

\subsection{Phidget Accelerometer/Magnetometer}

In order to be able to calibrate the tilt axis, we bought a separate electronics component that 
can measure 3 dimensional acceleration (and hence tilt).  Specifically, we bought the USB-connected
Phidget 1044.\footnote{\url{http://www.phidgets.com/products.php?product\_id=1044\_0}}



\section{MIDAS front-end Programs}

There are a number of different front-ends needed to run the PTF.  A summary of the front-end is
as follows:

\begin{itemize}
\item feMotor0[0,1]: This front-end talks to the Galil controller and provides motor control in 
low-level motor variables.
\item feMove: this front-end provides the user the ability to control the gantry in meaningful
coordinate system, with collision avoidance and initialization.  feMove initiates moves through
feMotor settings.
\item feScan: this front-end provides the ability to do a full gantry scan once a run is started.
feScan initiates moves through the feMove settings.
\item fePhidget0[0,1: these front-end logs the information from the Phidget accelerometers/magnetometers.
\end{itemize}

The source code for all front-ends is under SVN version control.
Details are given in following sections. 

\subsection{feMotor}



The first frontend, called feMotor, was written before the first project team began 
work.\footnote{Originally written by Dave Morris of controls group.}
Therefore, it contains much of the basic functionality for the control software. By itself, 
it can be used to set the motor velocity and acceleration of each axis, to send the motors 
to a relative position on each axis and to monitor real-time variables such as the current 
position, the current state of the limit switches and whether or not the motors are currently 
moving. 
All the feMotor settings are in low-level units of motor steps.
and has many options that will rarely be changed; the user should rarely have to change anything
in feMotor.
The user might occassionally want to check the feMotor variables to check if a limit switch 
has correctly activated.

There are two different feMotor front-ends, feMotor00 and feMotor01, each of which talks
to a single Galil controller.  Currently we only use feMotor00.

\subsection{feMotor}

feMove provides a meaningful control level to the user.
feMove does this by only communicating to those values which are necessary for movement. 
These can be separated into three categories: settings, controls and variables. 

\begin{itemize}
\item The 'settings' values in the ODB are read into feMove once upon initialization. 
These settings include velocity and acceleration as well as the additional settings of 
motor scaling, axis channels and limit positions. 

\item The control values are used to send 
commands to the axis motors. Therefore, they are hot-linked to functions in feMove which 
send the desired commands to feMotor. Hot-linking mean that, if the user changes the 
control values in the ODB, the function is called. These controls include Start Move 
and Stop Move as well as the additional control, ReInitialize. Furthermore, an array 
of destination values for the axes is included in the control values. This, however, 
is not hot-linked to any particular function and is simply read into feMotor when 
Start Move is set to true. 

\item The variables are used to monitor aspects of the Testing Rig in real-time. Like 
feMotor, they include the current position, the current state of the limit switches and 
whether or not the motors are currently moving. However, in addition to these, feMove 
also monitors whether or not a move has been completed, whether or 
not the gantry destinations have been swapped and whether or not the axes have been initialized. 
\end{itemize}

The initialization of the axes is used to
convert the relative position in counts that feMotor uses into absolute positions in meters. 
This is done by 
defining an origin using the position of the limit switches. 
Specifically, the initialization routine will drive each motor to its negative limit 
position; once it gets there the feMove has a reference point for that axis.\footnote{As noted above
the tilt axis doesn't have a limit switch; instead it use the Phidget tilt measurement to try
to level the axis.}
After the initialization feMove can meaningfully present the gantry positions in 
meters in an absolute coordinate system.\footnote{though not necessary the best coordinate
system. See final section of manual on known problems.}

This initialization is very important for the current usage of the PTF.  The initialization
must be rerun every time that the feMove program is started (because there is no way
to know where the gantry was left after feMove was last turned off).
If you try to initiate a move to a particular position before the initialization has been
run, then feMove will automatically run the initialization first.



\subsection{feScan}
	
The purpose of feScan is to string together the individual movements of feMove into a path.  
Currently feScan just performs a predefined set of scans around a point at the middle of the PTF area.
You initiate a scan by starting a MIDAS run.

If desired, feScan can 
also take Magnetic Field data by communicating with the fourth frontend, fedvm. fedvm is 
solely used to send commands to a digital voltmeter connected to a 3-axis Hall Probe. It 
is being used "as is" from another project which required similar data. feScan uses the 
"run" feature in MIDAS to produce Magnetic Field and position data in "banks". The format 
of these banks can be read by an auxiliary tool “analyzer\_example.exe”, which produces 
the data tables found in the Results section below.\textit{This feature is currently not 
working.}
	
\section{Instruction Checklist}

The following is a checklist for operating the PTF.

\begin{enumerate}
\item Inspect the PTF gantry area; ensure that all obstructions are removed
and that gantry motion is unimpeded.

\item Logon to midptf01; username 'midptf', password 'pmtCalib???'

\item Turn on the power to the Galil motor controllers (left side of electronics rack).

\item Open the MIDAS webpage interface by going to 

\url{http://midptf01:8081}

\item Turn on the feMotor, feMove, feScan and fePhidget front-end programs.  You can 
do this from the Programs link off the MIDAS status page (preferred method)
or you can start the front-ends from a terminal by doing

cd online/src

feMotor -i 0
feMotor -i 1

feMove

feScan

fePhidget -i 0 
fePhidget -i 1

The six last commands must be done in separate windows.  Starting the commands on the terminal 
is useful if you want to check additional debugging information.

\item Now you should initialize the gantry position in feMove.  You can do this through the feMove
ODB Control settings, but it is easier to go to the 'Gantry Control' custom page:

\url{http://midptf01:8081}

and click 'Reinitialize Gantry'.

The gantry initialization will take a couple minutes (depending on where the gantry starts).  You will
hear an audible click when the gantry reachs the limit switch of each of the three linear axes.  Keep an
eye on the gantry to ensure that the limits are all correctly reached.

\item You can then use the 'Destination' fields in the 'Gantry Control' page to choose a new location which
to move the gantry; then click 'Move Gantry' to initiate a move.  Remember there is no collision avoidance in the
feMove code yet!  Be careful!

\item Finally you can try running a scan.  As long as the feScan program is running, just 
click 'start' from the MIDAS status page and start a run.  The scan sequence takes ~15 minutes now.

\item Once you are finished for the day, power down the galil controller.  That seems like the safe practice.



\end{enumerate}

\section{List of Known Problems and Needed Improvements}

Work still to do:

\begin{enumerate}
\item The second gantry is not at all cabled.  It needs all cable and electronics installed.  This will presumably wait
until we are fully satisfied with the first gantry.

\item Don't have any idea about the cables for the laser, APD, etc in the housing.  Need to ensure that cable runs
 work for these cables as well.

\item Some of the Galil limit switches seem to intermittently fail; suspect that we need to switch to another set of Galil
channels.

\item Need to modify initialization so that we move fully up in Z before initializing in X and Y.   Wayne says that this
should ensure that we avoid the edge of the water volume.

\item Need to implement the collision avoidance code and give it a sensible set of objects to avoid.

\item Need to define a sensible global coordinate system for the user; for instance, +Z is currently pointing 
down.
\end{enumerate}


\end{document}
